% Options for packages loaded elsewhere
\PassOptionsToPackage{unicode}{hyperref}
\PassOptionsToPackage{hyphens}{url}
\PassOptionsToPackage{dvipsnames,svgnames,x11names}{xcolor}
%
\documentclass[
  11pt,
]{article}

\usepackage{amsmath,amssymb}
\usepackage{iftex}
\ifPDFTeX
  \usepackage[T1]{fontenc}
  \usepackage[utf8]{inputenc}
  \usepackage{textcomp} % provide euro and other symbols
\else % if luatex or xetex
  \usepackage{unicode-math}
  \defaultfontfeatures{Scale=MatchLowercase}
  \defaultfontfeatures[\rmfamily]{Ligatures=TeX,Scale=1}
\fi
\usepackage{lmodern}
\ifPDFTeX\else  
    % xetex/luatex font selection
    \setmainfont[]{Aptos}
\fi
% Use upquote if available, for straight quotes in verbatim environments
\IfFileExists{upquote.sty}{\usepackage{upquote}}{}
\IfFileExists{microtype.sty}{% use microtype if available
  \usepackage[]{microtype}
  \UseMicrotypeSet[protrusion]{basicmath} % disable protrusion for tt fonts
}{}
\makeatletter
\@ifundefined{KOMAClassName}{% if non-KOMA class
  \IfFileExists{parskip.sty}{%
    \usepackage{parskip}
  }{% else
    \setlength{\parindent}{0pt}
    \setlength{\parskip}{6pt plus 2pt minus 1pt}}
}{% if KOMA class
  \KOMAoptions{parskip=half}}
\makeatother
\usepackage{xcolor}
\setlength{\emergencystretch}{3em} % prevent overfull lines
\setcounter{secnumdepth}{-\maxdimen} % remove section numbering
% Make \paragraph and \subparagraph free-standing
\makeatletter
\ifx\paragraph\undefined\else
  \let\oldparagraph\paragraph
  \renewcommand{\paragraph}{
    \@ifstar
      \xxxParagraphStar
      \xxxParagraphNoStar
  }
  \newcommand{\xxxParagraphStar}[1]{\oldparagraph*{#1}\mbox{}}
  \newcommand{\xxxParagraphNoStar}[1]{\oldparagraph{#1}\mbox{}}
\fi
\ifx\subparagraph\undefined\else
  \let\oldsubparagraph\subparagraph
  \renewcommand{\subparagraph}{
    \@ifstar
      \xxxSubParagraphStar
      \xxxSubParagraphNoStar
  }
  \newcommand{\xxxSubParagraphStar}[1]{\oldsubparagraph*{#1}\mbox{}}
  \newcommand{\xxxSubParagraphNoStar}[1]{\oldsubparagraph{#1}\mbox{}}
\fi
\makeatother


\providecommand{\tightlist}{%
  \setlength{\itemsep}{0pt}\setlength{\parskip}{0pt}}\usepackage{longtable,booktabs,array}
\usepackage{calc} % for calculating minipage widths
% Correct order of tables after \paragraph or \subparagraph
\usepackage{etoolbox}
\makeatletter
\patchcmd\longtable{\par}{\if@noskipsec\mbox{}\fi\par}{}{}
\makeatother
% Allow footnotes in longtable head/foot
\IfFileExists{footnotehyper.sty}{\usepackage{footnotehyper}}{\usepackage{footnote}}
\makesavenoteenv{longtable}
\usepackage{graphicx}
\makeatletter
\def\maxwidth{\ifdim\Gin@nat@width>\linewidth\linewidth\else\Gin@nat@width\fi}
\def\maxheight{\ifdim\Gin@nat@height>\textheight\textheight\else\Gin@nat@height\fi}
\makeatother
% Scale images if necessary, so that they will not overflow the page
% margins by default, and it is still possible to overwrite the defaults
% using explicit options in \includegraphics[width, height, ...]{}
\setkeys{Gin}{width=\maxwidth,height=\maxheight,keepaspectratio}
% Set default figure placement to htbp
\makeatletter
\def\fps@figure{htbp}
\makeatother

\usepackage{booktabs}
\usepackage{longtable}
\usepackage{array}
\usepackage{multirow}
\usepackage{wrapfig}
\usepackage{float}
\usepackage{colortbl}
\usepackage{pdflscape}
\usepackage{tabu}
\usepackage{threeparttable}
\usepackage{threeparttablex}
\usepackage[normalem]{ulem}
\usepackage{makecell}
\usepackage{xcolor}
\usepackage{pdfpages}
\usepackage{fontspec}
\setmainfont{Aptos}[Path="C:/Users/ginow/AppData/Roaming/TinyTeX/texmf-dist/fonts/truetype/aptos/", Extension=".ttf"]
\usepackage{fancyhdr}
\pagestyle{fancy}
\fancyhf{}
\cfoot{\thepage}
\floatplacement{figure}{H}
\floatplacement{table}{H}
\usepackage{geometry}
\geometry{ left=3cm, right=3cm, top=2.5cm, bottom=2.5cm }
\usepackage{placeins}
\usepackage{ragged2e}
\usepackage{float}
\usepackage{setspace}
\renewcommand{\familydefault}{\sfdefault}
\AtBeginDocument{\renewcommand{\baselinestretch}{1.5}\justifying}
\usepackage{xcolor}
\definecolor{mybgcolor}{HTML}{01376B}
\usepackage{pagecolor}
\makeatletter
\@ifpackageloaded{caption}{}{\usepackage{caption}}
\AtBeginDocument{%
\ifdefined\contentsname
  \renewcommand*\contentsname{Tabla de contenidos}
\else
  \newcommand\contentsname{Tabla de contenidos}
\fi
\ifdefined\listfigurename
  \renewcommand*\listfigurename{Listado de Figuras}
\else
  \newcommand\listfigurename{Listado de Figuras}
\fi
\ifdefined\listtablename
  \renewcommand*\listtablename{Listado de Tablas}
\else
  \newcommand\listtablename{Listado de Tablas}
\fi
\ifdefined\figurename
  \renewcommand*\figurename{Figura}
\else
  \newcommand\figurename{Figura}
\fi
\ifdefined\tablename
  \renewcommand*\tablename{Tabla}
\else
  \newcommand\tablename{Tabla}
\fi
}
\@ifpackageloaded{float}{}{\usepackage{float}}
\floatstyle{ruled}
\@ifundefined{c@chapter}{\newfloat{codelisting}{h}{lop}}{\newfloat{codelisting}{h}{lop}[chapter]}
\floatname{codelisting}{Listado}
\newcommand*\listoflistings{\listof{codelisting}{Listado de Listados}}
\makeatother
\makeatletter
\makeatother
\makeatletter
\@ifpackageloaded{caption}{}{\usepackage{caption}}
\@ifpackageloaded{subcaption}{}{\usepackage{subcaption}}
\makeatother

\ifLuaTeX
\usepackage[bidi=basic]{babel}
\else
\usepackage[bidi=default]{babel}
\fi
\babelprovide[main,import]{spanish}
\ifPDFTeX
\else
\babelfont{rm}[]{Aptos}
\fi
% get rid of language-specific shorthands (see #6817):
\let\LanguageShortHands\languageshorthands
\def\languageshorthands#1{}
\ifLuaTeX
  \usepackage{selnolig}  % disable illegal ligatures
\fi
\usepackage{bookmark}

\IfFileExists{xurl.sty}{\usepackage{xurl}}{} % add URL line breaks if available
\urlstyle{same} % disable monospaced font for URLs
\hypersetup{
  pdflang={es},
  colorlinks=true,
  linkcolor={blue},
  filecolor={Maroon},
  citecolor={Blue},
  urlcolor={Blue},
  pdfcreator={LaTeX via pandoc}}


\author{}
\date{}

\begin{document}

\includepdf[pages=-]{../Quarto/Portada/Portadas-Enadel-2023-1.pdf}


\newpage

\pagecolor{mybgcolor} 
\color{white}

\centering

\includegraphics[width=0.5\textwidth]{../Logotipo ENADEL/Logotipo ENADEL 2023.png}
\vspace{2cm}

\noindent Ministerio del Trabajo y Previsión Social

División de Políticas de Empleo\textbackslash{} Subsecretaría del
Trabajo

\justifying

El presente documento analiza los resultados de la Encuesta Nacional de
Demanda Laboral (ENADEL) 2023, que busca identificar y caracterizar el
capital humano requerido por las empresas de los distintos sectores
productivos del país, generando información sobre la demanda actual de
ocupaciones de las empresas, detectando requisitos y problemas de
contratación. Al igual que versiones anteriores de esta encuesta, se
puso el foco en dos sectores de actividad económica: Construcción y el
sector Agrícola.

\newpage

\pagecolor{white} \% Volver al fondo blanco para el resto del documento
\color{black}

\newpage
\renewcommand{\contentsname}{Índice} 
\tableofcontents

\newpage

\subsection{Descripción General: Empresas y
Trabajadores}\label{descripciuxf3n-general-empresas-y-trabajadores}

La muestra de ENADEL 2023 encuesta a 5.820 empresas que suman 485.256
trabajadores (a nivel muestral). Estas representan a 82.052 empresas y
5.611.196 trabajadores a nivel nacional. La Tabla~\ref{tbl-region}
muestra la distribución en las distintas regiones del país, dónde un
\text{53,6}\% de las empresas y un \text{65,6}\% de los trabajadores se
encuentran en la región Metropolitana.

\vspace{5mm}

\FloatBarrier

\begin{table}

\caption{\label{tbl-region}Resultados de la encuesta}

\centering{

\centering
\begin{tabular}{lll}
\toprule
Región & \% Empresas & \% Trabajadores\\
\midrule
Arica y Parinacota & 0,67\% & 0,33\%\\
Tarapacá & 1,69\% & 1,07\%\\
Antofagasta & 2,6\% & 1,82\%\\
Atacama & 0,76\% & 0,54\%\\
Coquimbo & 2,78\% & 1,86\%\\
\addlinespace
Valparaíso & 9,05\% & 6,5\%\\
Metropolitana & 53,61\% & 65,59\%\\
O'Higgins & 5,1\% & 4,21\%\\
Maule & 1,54\% & 1,01\%\\
Ñuble & 6,66\% & 4,96\%\\
\addlinespace
Biobío & 1,35\% & 0,69\%\\
Araucanía & 4,59\% & 3,45\%\\
Los Ríos & 0,39\% & 0,25\%\\
Los Lagos & 0,98\% & 0,64\%\\
Aysén & 4,92\% & 4,74\%\\
\addlinespace
Magallanes & 3,3\% & 2,34\%\\
\bottomrule
\multicolumn{3}{l}{\rule{0pt}{1em}Fuente: Elaboración propia utilizando datos de ENADEL 2023, datos expandidos.}\\
\end{tabular}

}

\end{table}%

\FloatBarrier

La Figura~\ref{fig-combined} muestra el porcentaje de empresas y
trabajadores según tamaño de empresas, utilizando la clasificación por
número de trabajadores, cómo por volumen de venta. Con respecto a la
primera clasificación, el 74,6\% de las empresas tiene menos de 50
trabajadores --acumulando un 23,8\% del total-- y casi el 50\% de los
trabajadores están en empresas grandes --que corresponden a un 6,5\% del
total. Si se analiza según tamaño de ventas, más de la mitad de las
empresas tienen un volumen de venta de entre 2.400 y 24.999 UF
(``Pequeñas'') y más de un cuarto venden entre 25.000 y 100.000 UF al
año (``Mediana''). Sin embargo, casi un tercio de los trabajadores están
en empresas grandes (más de 100.000 UF).

\FloatBarrier

\begin{figure}[H]

\caption{\label{fig-combined}Gráfico combinado de resultados}

\centering{

\includegraphics[width=1\textwidth,height=\textheight]{Reporte_files/figure-pdf/fig-combined-1.pdf}

}

\end{figure}%

\FloatBarrier

Al revisar la distribución por sector de actividad económica
Tabla~\ref{tbl-acteco} se tiene que una de cada cinco empresas pertenece
al sector de Comercio, seguido por el sector Construcción
(\text{15,5}\%) y el sector de Industrias Manufactureras (11,4\%). Con
respecto al volumen de trabajadores, el sector Comercio también lidera
(17,8\%) seguido por el sector Construcción (14,4\%) y el sector de
Servicios Administrativos y de Apoyo que, siendo sector intensivo en
trabajo, un 9,1\% de las empresas acumula el 13,5\% de trabajadores y
trabajadoras.

\FloatBarrier

\begin{table}

\caption{\label{tbl-acteco}Empresas y trabajadores según sector de
actividad económica}

\centering{

\centering
\begin{tabular}{lll}
\toprule
Actividad Económica & \% Empresas & \% Trabajadores\\
\midrule
Comercio & 20,6\% & 17,78\%\\
Construcción & 15,54\% & 14,42\%\\
Servicios administrativos y de apoyo & 9,08\% & 13,5\%\\
Industria manufacturera & 11,39\% & 11,81\%\\
Actividades profesionales & 10,82\% & 10,97\%\\
\addlinespace
Silvoagropecuario & 8,64\% & 8,33\%\\
Transporte y almacenamiento & 7,62\% & 5,91\%\\
Administración pública & 0,69\% & 5,46\%\\
Alojamiento y de servicio de comidas & 6,79\% & 4,03\%\\
Información y comunicaciones & 3,02\% & 2,61\%\\
\addlinespace
Actividades inmobiliarias & 3,15\% & 2,6\%\\
Actividades financieras y de seguros & 1,88\% & 1,69\%\\
Pesca y acuicultura & 0,77\% & 0,9\%\\
\bottomrule
\multicolumn{3}{l}{\rule{0pt}{1em}Fuente: Elaboración propia utilizando datos de ENADEL 2023, datos expandidos.}\\
\end{tabular}

}

\end{table}%

\FloatBarrier

\subsubsection{Contratados en los últimos 12
meses}\label{contratados-en-los-uxfaltimos-12-meses}

El 87,1\% de las empresas declaró que sí contrató personas nuevas
durante los últimos 12 meses, mientras que el 12,9\% declara que no
contrató personas nuevas durante el mismo período. La
Tabla~\ref{tbl-contratos_totales} muestra la proporción de empresas,
según sector de actividad económica, que tuvieron contrataciones nuevas
durante los últimos 12 meses. El 98.9\% de las empresas del sector de
pesca y acuicultura respondieron afirmativamente, mientras que el sector
con menor proporción fue el de Actividades Inmobiliarias, dónde el
24,9\% de firmas no tuvieron contrataciones durante los últimos 12
meses.

\FloatBarrier

\begin{table}

\caption{\label{tbl-contratos_totales}Porcentaje de empresas que
contrataron en los últimos 12 meses por sector económico}

\centering{

\centering
\begin{tabular}{lll}
\toprule
Actividad Económica & \% Si & \% No\\
\midrule
Actividades profesionales & 90,4\% & 9,6\%\\
Actividades financieras y de seguros & 83,9\% & 16,1\%\\
Actividades inmobiliarias & 75,1\% & 24,9\%\\
Administración pública & 90,3\% & 9,7\%\\
Alojamiento y de servicio de comidas & 92,9\% & 7,1\%\\
\addlinespace
Comercio & 84,8\% & 15,2\%\\
Construcción & 89,3\% & 10,7\%\\
Industria manufacturera & 87,3\% & 12,7\%\\
Información y comunicaciones & 90,3\% & 9,7\%\\
Servicios administrativos y de apoyo & 87,5\% & 12,5\%\\
\addlinespace
Silvoagropecuario & 83,6\% & 16,4\%\\
Pesca y acuicultura & 98,9\% & 1,1\%\\
Transporte y almacenamiento & 85\% & 15\%\\
\bottomrule
\multicolumn{3}{l}{\rule{0pt}{1em}Fuente: Elaboración propia utilizando datos de ENADEL 2023, datos expandidos.}\\
\end{tabular}

}

\end{table}%

\FloatBarrier

\subparagraph{Contrataciones por
ocupación}\label{contrataciones-por-ocupaciuxf3n}

Se contrataron cargos\footnote{Contrataciones señaladas para los módulos
  de ``Directores y gerentes'', ``Jefatura/ADP'', ``Ocupaciones
  Elementales'' y ``Otros Cargos''.} nuevos en 354 ocupaciones
(representando a 1.206.992 contrataciones totales en los últimos 12
meses, cv = 6,8\%), sin embargo, la Tabla~\ref{tbl-contratados_u12} sólo
muestra aquellas para las cuáles el coeficiente de variación es menor a
20\%\footnote{La convención es considerar cómo robustas estimaciones con
  un cv menor al 15\%. Dado que estas son pocas, se presentan todas las
  ocupaciones con un cv menor a 20\%.}. Las ocupaciones con más
contratos en los últimos 12 meses fueron ``Obreros de explotaciones
agrícolas'', ``Auxiliares de aseo de oficinas, hoteles y otros
establecimientos'' y ``Vendedores y asistentes de venta de tiendas,
almacenes y puestos de mercado''.

\begin{table}

\caption{\label{tbl-contratados_u12}Contratados últimos 12 meses, por
ocupación.}

\centering{

\centering
\begin{tabular}{r>{\raggedright\arraybackslash}p{11cm}rl}
\toprule
CIUO\_08 & Glosa & Contratados & cv\\
\midrule
9211 & Obreros de explotaciones agrícolas & 300480 & 19,2\%\\
9112 & Auxiliares de aseo de oficinas, hoteles y otros establecimientos & 71821 & 16,4\%\\
5223 & Vendedores y asistentes de venta de tiendas, almacenes y puestos de mercado & 23492 & 15,7\%\\
5414 & Guardias de seguridad & 17719 & 12,9\%\\
5131 & Garzones de mesa & 13484 & 10,4\%\\
\addlinespace
8332 & Conductores de camiones pesados y de alto tonelaje & 12410 & 9,5\%\\
5120 & Cocineros & 7623 & 14,7\%\\
3123 & Supervisores de la construcción & 5224 & 19,2\%\\
2243 & Ingenieros en prevención de riesgos y otros profesionales de la seguridad e higiene laboral y ambiental & 2971 & 19,2\%\\
8331 & Conductores de buses y trolebuses & 2968 & 16,6\%\\
\addlinespace
8350 & Tripulantes de cubierta de barco & 1639 & 18,7\%\\
\bottomrule
\multicolumn{4}{l}{\rule{0pt}{1em}Fuente: Elaboración propia utilizando datos de ENADEL 2023, datos expandidos.}\\
\end{tabular}

}

\end{table}%

\newpage

\subparagraph{Contrataciones por ocupación: Sector
Construcción}\label{contrataciones-por-ocupaciuxf3n-sector-construcciuxf3n}

En el módulo del sector Construcción se consulta sobre las
contrataciones de 25 cargos específicos, con un total de 365.712
contrataciones en los últimos 12 meses (cv=29,2\%). La
Tabla~\ref{tbl-contratados_u12_constru} muestra la proporción de
contratados por cargo, con respecto al total de este módulo. El cargo de
``Obreros y jornales'' es el que concentra la mayor cantidad de
contrataciones, seguido por ``Carpinteros'' y por ``Electricistas
(técnicos y/o maestros)''.

\begin{table}

\caption{\label{tbl-contratados_u12_constru}Proporción de contrataciones
por ocupación, sector construcción.}

\centering{

\centering
\begin{tabular}{ll}
\toprule
Cargo Construcción & \%\\
\midrule
Obreros y jornales & 23,2\%\\
Carpinteros & 12,2\%\\
Electricistas (técnicos y/o maestros) & 11,5\%\\
Otros maestros de primera y segunda & 9,7\%\\
Albañiles & 7\%\\
\addlinespace
Enfierradores & 4\%\\
Concreteros & 3,7\%\\
Operadores de maquinaria pesada & 3,2\%\\
Pintores & 3,1\%\\
Operadores de maquinaria liviana & 3\%\\
\addlinespace
Ingenieros, prevencionistas, arqueólogos, otros profesionales & 2,6\%\\
Baldoseros y ceramistas & 2,6\%\\
Soldadores & 2,4\%\\
Capataces & 2,3\%\\
Encargados de Obra & 2,1\%\\
\addlinespace
Electrónicos, electromecánicos e instrumentistas & 2,1\%\\
Mecánicos & 1,6\%\\
Trazadores & 1,1\%\\
Bodegueros y cardcheckers & 1\%\\
Sanitarios y gásfiteres & 0,4\%\\
\addlinespace
Instaladores de gas & 0,4\%\\
Operadores planta asfalto y áridos & 0,3\%\\
Laboratoristas & 0,2\%\\
Tuberos y peradores de termofusión & 0,2\%\\
Buzos & 0\%\\
\bottomrule
\multicolumn{2}{l}{\rule{0pt}{1em}Fuente: Elaboración propia utilizando datos de ENADEL 2023, datos expandidos.}\\
\end{tabular}

}

\end{table}%

\subparagraph{Contrataciones por ocupación: Sector
Agrícola}\label{contrataciones-por-ocupaciuxf3n-sector-agruxedcola}

En el módulo del sector Agrícola se consulta sobre las contrataciones de
15 cargos específicos, con un total de 321.395 contrataciones en los
últimos 12 meses (cv=21,6\%), cuya distribución en los distintos cargos
se muestra en la Tabla~\ref{tbl-contratados_u12_agricola}. Las
ocupaciones de ``Obrero agrícola de cosecha'' y ``Obrero agrícola de
packing frutícola, bodega, estabilización, embotellado'' son las que
lideran las contrataciones durante los últimos 12 meses.

\begin{table}

\caption{\label{tbl-contratados_u12_agricola}Proporción de
contrataciones por ocupación, sector agrícola}

\centering{

\centering
\begin{tabular}{ll}
\toprule
Cargo Agrícola & \%\\
\midrule
Obrero agrícola de cosecha & 37,6\%\\
Obrero agrícola de packing frutícola, bodega, estabilización, embotellado & 37,4\%\\
Obrero agrícola de poda, raleo & 11,3\%\\
Obrero agrícola de siembra, viveros & 3,7\%\\
Obrero agrícola de riego, aplicación de agroquímicos & 3,2\%\\
\addlinespace
Obrero forestal de cosecha & 1,9\%\\
Obrero agrícola de almácigos & 1,7\%\\
Obrero forestal de siembra & 1,2\%\\
Obrero pecuario de crianza, alimentación, pastoreo & 0,8\%\\
Obrero agroindustrial gestión de cría y engorda & 0,6\%\\
\addlinespace
Obrero forestal en labores de aserrador & 0,3\%\\
Obrero pecuario de ordeña & 0,2\%\\
Obrero agrícola de manejo reproductivo y sanitario & 0\%\\
Obrero agroindustrial en matadero & 0\%\\
Obrero agroindustrial limpieza, control plagas y enfermedades & 0\%\\
\bottomrule
\multicolumn{2}{l}{\rule{0pt}{1em}Fuente: Elaboración propia utilizando datos de ENADEL 2023, datos expandidos.}\\
\end{tabular}

}

\end{table}%

\subsubsection{Vacantes en los últimos 12
meses}\label{vacantes-en-los-uxfaltimos-12-meses}

El 72,05\% de las empresas declaró que sí tuvo vacantes no llenadas
durante los últimos 12 meses, mientras que el 27,95\% declara que no
tuvo vacantes no llenadas durante el mismo período.

La Tabla~\ref{tbl-vacantes_totales_sector} muestra la proporción de
empresas, según sector de actividad económica, que tuvieron vacantes no
llenadas durante los últimos 12 meses. El 89.5\% de las empresas del
sector de ``Información y comunicaciones'' tuvieron vacantes no
llenadas, lo mismo ocurre con el 87.7\% de las personas jurídicas del
sector de ``Administración pública'' y el 80.2\% del sector de
``Actividades profesionales''. Por otro lado, el sector de actividad
económica con un menor porcentaje de empresas que tuvieron vacantes no
llenadas fue el ``Silvoagropecuario''.

\begin{table}

\caption{\label{tbl-vacantes_totales_sector}Porcentaje de empresas con
vacantes no cubiertas}

\centering{

\centering
\begin{tabular}{lll}
\toprule
Sector Actividad Económica & \% Sí & \% No\\
\midrule
Información y comunicaciones & 89,5\% & 10,5\%\\
Administración pública & 87,7\% & 12,3\%\\
Actividades profesionales & 80,2\% & 19,8\%\\
Alojamiento y de servicio de comidas & 78,3\% & 21,7\%\\
Transporte y almacenamiento & 78,3\% & 21,7\%\\
\addlinespace
Servicios administrativos y de apoyo & 76,4\% & 23,6\%\\
Industria manufacturera & 73,7\% & 26,3\%\\
Comercio & 72,3\% & 27,7\%\\
Construcción & 67,8\% & 32,2\%\\
Actividades financieras y de seguros & 67,6\% & 32,4\%\\
\addlinespace
Actividades inmobiliarias & 63\% & 37\%\\
Pesca y acuicultura & 61,7\% & 38,3\%\\
Silvoagropecuario & 49,4\% & 50,6\%\\
\bottomrule
\multicolumn{3}{l}{\rule{0pt}{1em}Fuente: Elaboración propia utilizando datos de ENADEL 2023, datos expandidos.}\\
\end{tabular}

}

\end{table}%

Durante el resto de esta sección se hará referencia a las ocupaciones
que tengan vacantes sin llenar durante los últimos 12 meses como
\textbf{ocupaciones de difícil cobertura}.

\newpage

\subparagraph{Ocupaciones de difícil
cobertura}\label{ocupaciones-de-difuxedcil-cobertura}

Se declararon vacantes\footnote{Vacantes señaladas para los módulos de
  ``Directores y gerentes'', ``Jefatura/ADP'', ``Ocupaciones
  Elementales'' y ``Otros Cargos''.} sin llenar en 215 ocupaciones
(68.365 vacantes, cv=12,3\%), sin embargo, la
Tabla~\ref{tbl-vacantes_tot} sólo muestra aquellas para las cuáles el
coeficiente de variación es menor a 40\%\footnote{La convención es
  considerar cómo robustas estimaciones con un cv menor al 15\%. Dado
  que esto no se cumple, se presentan todas las ocupaciones con un cv
  menor a 40\%.}. La tabla completa con todas las ocupaciones de difícil
cobertura se puede encontrar en el \textbf{Apéndice A: Ocupaciones de
Difícil Cobertura}.

\begin{table}

\caption{\label{tbl-vacantes_tot}Ocupaciones de difícil cobertura,
ENADEL 2023.}

\centering{

\centering
\begin{tabular}{rlrl}
\toprule
CIUO 08 & Glosa & Vacantes & cv\\
\midrule
9211 & Obreros de explotaciones agrícolas & 14374 & 38,3\%\\
5414 & Guardias de seguridad & 1193 & 39,4\%\\
9112 & Auxiliares de aseo de oficinas, hoteles y otros establecimientos & 1022 & 36,6\%\\
5223 & Vendedores y asistentes de venta de tiendas, almacenes y puestos de mercado & 476 & 35,2\%\\
5120 & Cocineros & 337 & 38,6\%\\
\addlinespace
8332 & Conductores de camiones pesados y de alto tonelaje & 232 & 33,8\%\\
\bottomrule
\multicolumn{4}{l}{\rule{0pt}{1em}Fuente: Elaboración propia utilizando datos de ENADEL 2023, datos expandidos.}\\
\end{tabular}

}

\end{table}%

La ocupación más demandadas que tiene difícil cobertura es ``Obreros de
explotaciones agrícolas'', con más de 14 mil vacantes; seguido de lejos
por ``Guardias de seguridad'', ``Auxiliares de aseo de oficinas, hoteles
y otros establecimientos'' y ``Vendedores y asistentes de venta de
tiendas, almacenes y puestos de mercado'', con 1193, 1022 y 476
vacantes, respectivamente.

La Tabla~\ref{tbl-vacantes_region} muestra la ocupación de difícil
cobertura con mayor cantidad de vacantes por región dónde, si bien se
confirma la prevalencia de ``Obreros de explotaciones agrícolas'' en 7
regiones, también saltan a la vista otros patrones como la demanda del
rubro de la construcción hacia el norte y de la industria manufacturera
en el Maule y la Araucanía.

\begin{table}

\caption{\label{tbl-vacantes_region}Ocupación de difícil cobertura con
mayor cantidad de vacantes, por región.}

\centering{

\centering
\begin{tabular}{l>{\raggedright\arraybackslash}p{7cm}rr}
\toprule
Región & Ocupación & CIUO 08 & Vacantes\\
\midrule
Arica y Parinacota & Obreros de la construcción de edificios & 9313 & 91\\
Tarapacá & Otros operarios de la construcción (obra gruesa) no clasificados previamente & 7119 & 112\\
Antofagasta & Soldadores y oxicortadores & 7212 & 66\\
Atacama & Obreros de explotaciones agrícolas & 9211 & 2943\\
Coquimbo & Cocineros de comida rápida & 9411 & 236\\
\addlinespace
Valparaíso & Obreros de explotaciones agrícolas & 9211 & 1076\\
Metropolitana & Obreros de explotaciones agrícolas & 9211 & 3376\\
O'Higgins & Secretarios generales & 4120 & 269\\
Maule & Obreros de la industria manufacturera no clasificados previamente & 9329 & 93\\
Ñuble & Agricultores y trabajadores calificados de cultivos extensivos & 6111 & 151\\
\addlinespace
Biobío & Obreros de explotaciones agrícolas & 9211 & 27\\
Araucanía & Operadores de máquinas de preparación de fibras, hilado y devanado & 8151 & 32\\
Los Ríos & Obreros de explotaciones agrícolas & 9211 & 7397\\
Los Lagos & Obreros de explotaciones agrícolas & 9211 & 49\\
Aysén & Auxiliares de aseo de oficinas, hoteles y otros establecimientos & 9112 & 41\\
\addlinespace
Magallanes & Obreros de explotaciones agrícolas & 9211 & 217\\
\bottomrule
\multicolumn{4}{l}{\rule{0pt}{1em}Fuente: Elaboración propia utilizando datos de ENADEL 2023, datos expandidos.}\\
\end{tabular}

}

\end{table}%

Al indagar según el tamaño de empresa (Tabla~\ref{tbl-vacantes_tamano}),
``Obreros de explotaciones agrícolas'' prevalece en las empresas
pequeñas (tanto por número de trabajadores como por tamaño de venta) y
``Empacadores manuales'' en las empresas con más de 200 trabajadores y
en aquellas con ventas entre 25 y 100 mil UF. La ocupación
``Especialistas en políticas y servicios de personal'' es la más difícil
de cubrir en empresas de entre 50 y 199 trabajadores (mediana) y en
aquellas que tienen ventas superiores a las 10 mil UF.

\begin{table}

\caption{\label{tbl-vacantes_tamano}Ocupación de difícil cobertura con
mayor cantidad de vacantes por tamaño de empresa.}

\centering{

\centering
\begin{tabular}{>{\raggedright\arraybackslash}p{4cm}l>{\raggedright\arraybackslash}p{5cm}>{\raggedleft\arraybackslash}p{2cm}>{\raggedleft\arraybackslash}p{2cm}}
\toprule
Categoría & Tamaño Empresa & Ocupación & CIUO 08 & Vacantes\\
\midrule
\addlinespace[0.3em]
\multicolumn{5}{l}{\textbf{\textbf{Según n° de trabajadores}}}\\
\hspace{1em}\textbf{} & Pequeña (<50) & Obreros de explotaciones agrícolas & 9211 & 12571\\
\hspace{1em}\textbf{} & Mediana (50-199) & Especialistas en políticas y servicios de personal & 2423 & 2459\\
\hspace{1em}\textbf{} & Grande (>=200) & Empacadores manuales & 9321 & 1547\\
\addlinespace[0.3em]
\multicolumn{5}{l}{\textbf{\textbf{Según ventas}}}\\
\hspace{1em}\textbf{} & Sin ventas & Otro personal de los servicios de protección no clasificados previamente & 5419 & 146\\
\hspace{1em}\textbf{} & Micro (< 2.400 UF) & Conductores de camiones pesados y de alto tonelaje & 8332 & 219\\
\hspace{1em}\textbf{} & Pequeña (2.400-25.000 UF) & Obreros de explotaciones agrícolas & 9211 & 14374\\
\hspace{1em}\textbf{} & Mediana (25.000-100.000 UF) & Empacadores manuales & 9321 & 1778\\
\hspace{1em}\textbf{} & Grande (>100.000 UF) & Especialistas en políticas y servicios de personal & 2423 & 2459\\
\bottomrule
\multicolumn{5}{l}{\rule{0pt}{1em}Fuente: Elaboración propia utilizando datos de ENADEL 2023, datos expandidos.}\\
\end{tabular}

}

\end{table}%

Al analizar las ocupaciones de difícil cobertura según el sector de
actividad económica, como se muestra en la
Tabla~\ref{tbl-vacantes_sector}, se vuelve a confirmar la dificultad
para llenar vacantes de ``Obreros de explotaciones agrícolas'', pero
también surgen otras ocupaciones relevantes como ``Obreros de
explotaciones agrícolas'' ``Vendedores y asistentes de venta de tiendas,
almacenes y puestos de mercado'' que es la ocupación con más vacantes
sin llenar en los sectores de ``Servicios administrativos y de apoyo'' y
``Transporte y almacenamiento''.

\begin{table}

\caption{\label{tbl-vacantes_sector}Ocupación de difícil cobertura con
mayor cantidad de vacantes por sector de actividad económica.}

\centering{

\centering
\begin{tabular}{l>{\raggedright\arraybackslash}p{5cm}>{\raggedleft\arraybackslash}p{3cm}>{\raggedleft\arraybackslash}p{3cm}}
\toprule
Sector Actividad Económica & Ocupación & CIUO 08 & Vacantes\\
\midrule
Silvoagropecuario & Obreros de explotaciones agrícolas & 9211 & 8654\\
Servicios administrativos y de apoyo & Obreros de explotaciones agrícolas & 9211 & 5755\\
Actividades profesionales & Especialistas en políticas y servicios de personal & 2423 & 2459\\
Alojamiento y de servicio de comidas & Ayudantes de cocina & 9412 & 1380\\
Construcción & Soldadores y oxicortadores & 7212 & 1377\\
\addlinespace
Transporte y almacenamiento & Obreros de carga & 9333 & 974\\
Comercio & Vendedores y asistentes de venta de tiendas, almacenes y puestos de mercado & 5223 & 882\\
Industria manufacturera & Obreros de la industria manufacturera no clasificados previamente & 9329 & 877\\
Información y comunicaciones & Instaladores y reparadores en tecnología de la información y las comunicaciones & 7422 & 340\\
Administración pública & Otro personal de los servicios de protección no clasificados previamente & 5419 & 146\\
\addlinespace
Actividades financieras y de seguros & Vendedores y asistentes de venta de tiendas, almacenes y puestos de mercado & 5223 & 43\\
Pesca y acuicultura & Carniceros y pescaderos & 7511 & 27\\
Actividades inmobiliarias & Personal de pompas fúnebres y embalsamadores & 5163 & 17\\
\bottomrule
\multicolumn{4}{l}{\rule{0pt}{1em}Fuente: Elaboración propia utilizando datos de ENADEL 2023, datos expandidos. Nota: para seleccionar la ocupación de difícil cobertura con mayor cantidad de vacantes sólo se tomó en cuenta la magnitud de la estimación, sin considerar indicadores de robustez cómo el coeficiente de variación.}\\
\end{tabular}

}

\end{table}%

\newpage

\subparagraph{Ocupaciones de Difícil Cobertura: Sector
Construcción}\label{ocupaciones-de-difuxedcil-cobertura-sector-construcciuxf3n}

La Tabla~\ref{tbl-vacantes_constru} muestra todos los cargos del sector
Construcción sobre los que se consulta en la encuesta, indicando el
número de vacantes, el porcentaje que estas representan del total, y el
coeficiente de variación. Se muestran todas las vacantes independiente
de la robustez estadística de la estimación.

La ocupación de ``Obreros y jornales'' es la más demandada, con 1271
vacantes, seguida de ``Carpinteros'' con 495 vacantes y ``Soldadores''
con 406 vacantes.

\begin{table}

\caption{\label{tbl-vacantes_constru}Ocupaciones de difícil cobertura,
sector construcción.}

\centering{

\centering
\begin{tabular}{>{\raggedright\arraybackslash}p{3cm}>{\raggedleft\arraybackslash}p{5cm}ll}
\toprule
Cargo Construcción & Vacantes & \% del total & CV\\
\midrule
Obreros y jornales & 1271 & 39\% & 46\%\\
Carpinteros & 495 & 15\% & 52\%\\
Soldadores & 406 & 13\% & 83\%\\
Otros maestros de primera y segunda & 184 & 6\% & 90\%\\
Ingenieros, prevencionistas, arqueólogos, otros profesionales & 137 & 4\% & 54\%\\
\addlinespace
Pintores & 131 & 4\% & 59\%\\
Enfierradores & 119 & 4\% & 58\%\\
Albañiles & 101 & 3\% & 44\%\\
Capataces & 99 & 3\% & 51\%\\
Encargados de Obra & 63 & 2\% & 50\%\\
\addlinespace
Electrónicos, electromecánicos e instrumentistas & 59 & 2\% & 60\%\\
Baldoseros y ceramistas & 54 & 2\% & 78\%\\
Electricistas (técnicos y/o maestros) & 42 & 1\% & 47\%\\
Operadores de maquinaria pesada & 34 & 1\% & 77\%\\
Bodegueros y cardcheckers & 12 & 0\% & 72\%\\
\addlinespace
Mecánicos & 6 & 0\% & 100\%\\
Laboratoristas & 4 & 0\% & 100\%\\
Sanitarios y gásfiteres & 3 & 0\% & 100\%\\
Total & 3220 & 99\% & NA\%\\
\bottomrule
\multicolumn{4}{l}{\rule{0pt}{1em}Fuente: Elaboración propia utilizando datos de ENADEL 2023, datos expandidos.}\\
\end{tabular}

}

\end{table}%

\newpage

\subparagraph{Ocupaciones de Difícil Cobertura: Sector
Agrícola}\label{ocupaciones-de-difuxedcil-cobertura-sector-agruxedcola}

La Tabla~\ref{tbl-vacantes_agro} muestra todos los cargos del sector
Agrícola sobre los que se consulta en la encuesta, indicando el número
de vacantes, el porcentaje que estas representan del total, y el
coeficiente de variación. Se muestran todas las vacantes independiente
de la robustez estadística de la estimación.

El cargo ``Obrero agrícola de poda, raleo'' es el más demandado, con
``2234'' vacantes, el segundo cargo más demandado es ``Obrero agrícola
de cosecha'' con 1921 vacantes y `` Obrero agrícola de riego, aplicación
de agroquímicos'' es el tercer cargo más demandado con 1597 vacantes.

\begin{table}

\caption{\label{tbl-vacantes_agro}Ocupaciones de difícil cobertura,
sector agrícola.}

\centering{

\centering
\begin{tabular}{>{\raggedright\arraybackslash}p{7cm}rll}
\toprule
Cargo Agrícola & Vacantes & \% del total & CV\\
\midrule
Obrero agrícola de poda, raleo & 2234 & 21\% & 70\%\\
Obrero agrícola de cosecha & 1921 & 18\% & 80\%\\
Obrero agrícola de riego, aplicación de agroquímicos & 1597 & 15\% & 93\%\\
Obrero agrícola de siembra, viveros & 1488 & 14\% & 100\%\\
Obrero agrícola de packing frutícola, bodega, estabilización, embotellado & 1488 & 14\% & 84\%\\
\addlinespace
Obrero agrícola de almácigos & 1488 & 14\% & 100\%\\
Obrero forestal de siembra & 196 & 2\% & 100\%\\
Obrero forestal de cosecha & 173 & 2\% & 82\%\\
Obrero pecuario de ordeña & 41 & 0\% & 100\%\\
Total & 10626 & 100\% & NA\%\\
\bottomrule
\multicolumn{4}{l}{\rule{0pt}{1em}Fuente: Elaboración propia utilizando datos de ENADEL 2023, datos expandidos.}\\
\end{tabular}

}

\end{table}%

Al comparar los sectores agrícola y construcción, se puede notar que el
primer sector tiene una mayor dificultad para llenar sus vacantes:
mientras en el sector construcción las vacantes difíciles de llenar
equivalen al 0,88\% del total de contrataciones en los últimos 12 meses;
en el sector agrícola equivalen al 3,3\% de las contrataciones.

\newpage

\subsubsection{Dificultades para la
contratación}\label{dificultades-para-la-contrataciuxf3n}

El 20,6\% de las empresas declaró tener alguna dificultad durante el
proceso de contratación para llenar las vacantes disponibles\footnote{Se
  incluyen todas las vacantes para todos los cargos: módulos de
  ``Directores y gerentes'', ``Jefatura/ADP'', ``Ocupaciones
  Elementales'' y ``Otros Cargos''; Construcción y Agrícola.}. La
Tabla~\ref{tbl-dificultad_tot} muestra la proporción de respuestas para
la primera dificultad mencionada y para el total. Del total de
respuestas, la dificultad señalada más frecuentemente es ``Condiciones
laborales no aceptadas'' (16.6 \%), seguida de ``Falta de postulantes''
(14.8 \%) y Rotación laboral'' (12.9 \%).

\begin{table}

\caption{\label{tbl-dificultad_tot}Dificultades principales de
contratación.}

\centering{

\centering
\begin{tabular}{>{\raggedright\arraybackslash}p{5cm}>{\raggedright\arraybackslash}p{3cm}>{\raggedright\arraybackslash}p{3cm}}
\toprule
Primera dificultad & \% de 1ra dificultad & \% del total\\
\midrule
Condiciones laborales no aceptadas & 15,5\% & 16,6\%\\
Falta de postulantes & 15\% & 14,8\%\\
Rotación laboral & 13,9\% & 12,9\%\\
Candidatos sin competencias técnicas requeridas & 17\% & 12,5\%\\
Remuneración ofrecida no aceptada & 12,1\% & 12,3\%\\
\addlinespace
Candidatos sin la experiencia mínima requerida & 8,2\% & 10\%\\
Otra dificultad & 5,6\% & 7,6\%\\
Candidatos sin competencias socioemocionales requeridas & 6,6\% & 7\%\\
Candidatos sin licencias, certificaciones o requisitos legales & 5,4\% & 4,8\%\\
Candidatos sin nivel educacional requerido & 0,7\% & 1,5\%\\
\bottomrule
\multicolumn{3}{l}{\rule{0pt}{1em}Fuente: Elaboración propia utilizando datos de ENADEL 2023, datos expandidos.}\\
\end{tabular}

}

\end{table}%

\subparagraph{Dificultades para la contratación: Sector
Construcción}\label{dificultades-para-la-contrataciuxf3n-sector-construcciuxf3n}

Tabla~\ref{tbl-dificultad_constru} muestra que, al considerar el total
de menciones, ``Remuneración ofrecida no aceptada'' (22.4 \%), fue la
dificultad más mencionada, seguida de ``Condiciones laborales no
aceptadas'' (20.4 \%) y Falta de postulantes'' (15.9 \%).

\begin{table}

\caption{\label{tbl-dificultad_constru}Dificultades principales de
contratación, ocupaciones del sector Construcción.}

\centering{

\centering
\begin{tabular}{>{\raggedright\arraybackslash}p{5cm}>{\raggedright\arraybackslash}p{3cm}>{\raggedright\arraybackslash}p{3cm}}
\toprule
Primera dificultad & \% de 1ra dificultad & \% del total\\
\midrule
Remuneración ofrecida no aceptada & 29,8\% & 22,4\%\\
Condiciones laborales no aceptadas & 8,4\% & 20,4\%\\
Falta de postulantes & 22\% & 15,9\%\\
Rotación laboral & 10\% & 14,7\%\\
Candidatos sin la experiencia mínima requerida & 7,7\% & 7,6\%\\
\addlinespace
Candidatos sin licencias, certificaciones o requisitos legales & 10,4\% & 5,9\%\\
Candidatos sin competencias técnicas requeridas & 7,8\% & 5,7\%\\
Candidatos sin nivel educacional requerido & 0,7\% & 3,8\%\\
Candidatos sin competencias socioemocionales requeridas & 2,9\% & 2,8\%\\
Otra dificultad & 0,2\% & 0,8\%\\
\bottomrule
\multicolumn{3}{l}{\rule{0pt}{1em}Fuente: Elaboración propia utilizando datos de ENADEL 2023, datos expandidos.}\\
\end{tabular}

}

\end{table}%

\subparagraph{Dificultades para la contratación: Sector
Agrícola}\label{dificultades-para-la-contrataciuxf3n-sector-agruxedcola}

El 17,6\% de las empresas consultadas sobre ocupaciones del sector
Agrícola declararon tener alguna dificultad durante el proceso de
contratación. La Tabla~\ref{tbl-dificultad_agro} indica que las tres
dificultades más mencionadas son ``Condiciones laborales no aceptadas''
(31.1 \%), ``Rotación laboral'' (20.9 \%) y Falta de postulantes'' (19.1
\%).

\begin{table}

\caption{\label{tbl-dificultad_agro}Dificultades principales de
contratación, ocupaciones del sector Construcción.}

\centering{

\centering
\begin{tabular}{>{\raggedright\arraybackslash}p{5cm}>{\raggedright\arraybackslash}p{3cm}>{\raggedright\arraybackslash}p{3cm}}
\toprule
Primera dificultad & \% de 1ra dificultad & \% del total\\
\midrule
Condiciones laborales no aceptadas & 40,1\% & 31,1\%\\
Rotación laboral & 14,8\% & 20,9\%\\
Falta de postulantes & 11,6\% & 19,1\%\\
Remuneración ofrecida no aceptada & 9\% & 8,2\%\\
Candidatos sin competencias técnicas requeridas & 11,6\% & 8,1\%\\
\addlinespace
Candidatos sin competencias socioemocionales requeridas & 5,6\% & 6\%\\
Candidatos sin la experiencia mínima requerida & 2,7\% & 3,4\%\\
Otra dificultad & 3,7\% & 2,6\%\\
Candidatos sin licencias, certificaciones o requisitos legales & 0,6\% & 0,4\%\\
Candidatos sin nivel educacional requerido & 0,2\% & 0,1\%\\
\bottomrule
\multicolumn{3}{l}{\rule{0pt}{1em}Fuente: Elaboración propia utilizando datos de ENADEL 2023, datos expandidos.}\\
\end{tabular}

}

\end{table}%

La Figura~\ref{fig-dificultad} compara la frecuencia de cada dificultad,
para la primera dificultad señalada, según el módulo de la encuesta. Si
bien ``Candidatos sin competencias técnicas requeridas'' es el primer
lugar de primeras menciones en todos los módulos de la encuesta, la
proporción en el sector agrícola es 1,3 veces más grande que en
construcción y 2,2 veces que en el módulo general. ``Condiciones
laborales no aceptadas'' es la segunda mayoría en los tres módulos, pero
es notoriamente más prevalente en construcción.

\begin{figure}[H]

\caption{\label{fig-dificultad}Primera dificultad señalada según módulo}

\centering{

\includegraphics[width=1\textwidth,height=\textheight]{Reporte_files/figure-pdf/fig-dificultad-1.pdf}

}

\end{figure}%

\newpage

\subsubsection{Educación y
experiencia}\label{educaciuxf3n-y-experiencia}

\subsubsection{Canales de reclutamiento}\label{canales-de-reclutamiento}

\subsubsection{Conocimiento de instituciones y
programas}\label{conocimiento-de-instituciones-y-programas}

\subsubsection{Apéndice A: Ocupaciones de Difícil
Cobertura}\label{apuxe9ndice-a-ocupaciones-de-difuxedcil-cobertura}




\end{document}
