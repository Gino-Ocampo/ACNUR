% Options for packages loaded elsewhere
\PassOptionsToPackage{unicode}{hyperref}
\PassOptionsToPackage{hyphens}{url}
%
\documentclass[
]{article}
\usepackage{amsmath,amssymb}
\usepackage{iftex}
\ifPDFTeX
  \usepackage[T1]{fontenc}
  \usepackage[utf8]{inputenc}
  \usepackage{textcomp} % provide euro and other symbols
\else % if luatex or xetex
  \usepackage{unicode-math} % this also loads fontspec
  \defaultfontfeatures{Scale=MatchLowercase}
  \defaultfontfeatures[\rmfamily]{Ligatures=TeX,Scale=1}
\fi
\usepackage{lmodern}
\ifPDFTeX\else
  % xetex/luatex font selection
\fi
% Use upquote if available, for straight quotes in verbatim environments
\IfFileExists{upquote.sty}{\usepackage{upquote}}{}
\IfFileExists{microtype.sty}{% use microtype if available
  \usepackage[]{microtype}
  \UseMicrotypeSet[protrusion]{basicmath} % disable protrusion for tt fonts
}{}
\makeatletter
\@ifundefined{KOMAClassName}{% if non-KOMA class
  \IfFileExists{parskip.sty}{%
    \usepackage{parskip}
  }{% else
    \setlength{\parindent}{0pt}
    \setlength{\parskip}{6pt plus 2pt minus 1pt}}
}{% if KOMA class
  \KOMAoptions{parskip=half}}
\makeatother
\usepackage{xcolor}
\usepackage[margin=1in]{geometry}
\usepackage{graphicx}
\makeatletter
\def\maxwidth{\ifdim\Gin@nat@width>\linewidth\linewidth\else\Gin@nat@width\fi}
\def\maxheight{\ifdim\Gin@nat@height>\textheight\textheight\else\Gin@nat@height\fi}
\makeatother
% Scale images if necessary, so that they will not overflow the page
% margins by default, and it is still possible to overwrite the defaults
% using explicit options in \includegraphics[width, height, ...]{}
\setkeys{Gin}{width=\maxwidth,height=\maxheight,keepaspectratio}
% Set default figure placement to htbp
\makeatletter
\def\fps@figure{htbp}
\makeatother
\setlength{\emergencystretch}{3em} % prevent overfull lines
\providecommand{\tightlist}{%
  \setlength{\itemsep}{0pt}\setlength{\parskip}{0pt}}
\setcounter{secnumdepth}{5}
\usepackage{fontspec}
\setmainfont{Century Gothic}
\usepackage{fancyhdr}
\usepackage{graphicx}
\usepackage{lastpage}
\setlength{\headheight}{12pt}
\fancypagestyle{plain}{ \fancyhf{} \lhead{ANÁLISIS ENADEL 2023} \rfoot{\thepage \hspace{1pt} de \pageref{LastPage}} }
\pagestyle{plain}
\usepackage{titling}
\usepackage{booktabs}
\usepackage{longtable}
\usepackage{array}
\usepackage{multirow}
\usepackage{wrapfig}
\usepackage{float}
\usepackage{colortbl}
\usepackage{pdflscape}
\usepackage{tabu}
\usepackage{threeparttable}
\usepackage{threeparttablex}
\usepackage[normalem]{ulem}
\usepackage{makecell}
\usepackage{xcolor}
\ifLuaTeX
  \usepackage{selnolig}  % disable illegal ligatures
\fi
\usepackage{bookmark}
\IfFileExists{xurl.sty}{\usepackage{xurl}}{} % add URL line breaks if available
\urlstyle{same}
\hypersetup{
  pdftitle={Análisis ENADEL 2023},
  pdfauthor={Cliodinámica},
  hidelinks,
  pdfcreator={LaTeX via pandoc}}

\title{Análisis ENADEL 2023}
\usepackage{etoolbox}
\makeatletter
\providecommand{\subtitle}[1]{% add subtitle to \maketitle
  \apptocmd{\@title}{\par {\large #1 \par}}{}{}
}
\makeatother
\subtitle{ACNUR}
\author{Cliodinámica}
\date{2024-10-08}

\begin{document}
\maketitle

\begin{titlingpage}
    \centering
    \includegraphics[width=0.5\textwidth]{Imagenes/color_MinTrabajo.png}
    \vspace{1cm} % Ajusta el espacio entre la imagen y el título

    \maketitle % Analisis ENADEL 2023

    \vspace{2cm} % Ajusta este espacio según sea necesario
\end{titlingpage}

Ministerio del Trabajo y Previsión Social

División de Políticas de Empleo Subsecretaría del Trabajo

El presente documento analiza los resultados de la Encuesta Nacional de
Demanda Laboral (ENADEL) 2023, que busca identificar y caracterizar el
capital humano requerido por las empresas de los distintos sectores
productivos del país, generando información sobre la demanda actual de
ocupaciones de las empresas, detectando requisitos y problemas de
contratación. Al igual que versiones anteriores de esta encuesta, se
puso el foco en dos sectores de actividad económica: Construcción y el
sector Agrícola.

\subsubsection{Descripción General: Empresas y
Trabajadores}\label{descripciuxf3n-general-empresas-y-trabajadores}

La muestra de ENADEL 2023 encuesta a 5.820 empresas que suman 485.256
trabajadores (a nivel muestral). Estas representan a 82.052 empresas y
5.611.196 trabajadores a nivel nacional. La Tabla 1 muestra la
distribución en las distintas regiones del país, dónde un 53.6\% de las
empresas y un 65.6\% de los trabajadores se encuentran en la región
Metropolitana.

\begin{longtable}[t]{lrr}
\caption{\label{tab:unnamed-chunk-2}Distribución de trabajadores y empresas por región, ENADEL 2023}\\
\toprule
Región & \% Empresas & \% Trabajadores\\
\midrule
Arica y Parinacota & 0.67 & 0.33\\
Tarapacá & 1.69 & 1.07\\
Antofagasta & 2.60 & 1.82\\
Atacama & 0.76 & 0.54\\
Coquimbo & 2.78 & 1.86\\
\addlinespace
Valparaíso & 9.05 & 6.50\\
Metropolitana & 53.61 & 65.59\\
O'Higgins & 5.10 & 4.21\\
Maule & 1.54 & 1.01\\
Ñuble & 6.66 & 4.96\\
\addlinespace
Biobío & 1.35 & 0.69\\
Araucanía & 4.59 & 3.45\\
Los Ríos & 0.39 & 0.25\\
Los Lagos & 0.98 & 0.64\\
Aysén & 4.92 & 4.74\\
\addlinespace
Magallanes & 3.30 & 2.34\\
\bottomrule
\multicolumn{3}{l}{\rule{0pt}{1em}\textit{Fuente: Elaboración propia utilizando datos de ENADEL 2023, datos expandidos.}}\\
\multicolumn{3}{l}{\rule{0pt}{1em}}\\
\end{longtable}

\subsubsection{Contratados en los últimos 12
meses}\label{contratados-en-los-uxfaltimos-12-meses}

\subparagraph{Contrataciones por
ocupación}\label{contrataciones-por-ocupaciuxf3n}

\subparagraph{Contrataciones por ocupación: Sector
Construcción}\label{contrataciones-por-ocupaciuxf3n-sector-construcciuxf3n}

\subparagraph{Contrataciones por ocupación: Sector
Agrícola}\label{contrataciones-por-ocupaciuxf3n-sector-agruxedcola}

\subsubsection{Vacantes en los últimos 12
meses}\label{vacantes-en-los-uxfaltimos-12-meses}

\subparagraph{Ocupaciones de difícil
cobertura}\label{ocupaciones-de-difuxedcil-cobertura}

\subparagraph{Ocupaciones de Difícil Cobertura: Sector
Construcción}\label{ocupaciones-de-difuxedcil-cobertura-sector-construcciuxf3n}

\subparagraph{Ocupaciones de Difícil Cobertura: Sector
Agrícola}\label{ocupaciones-de-difuxedcil-cobertura-sector-agruxedcola}

\subsubsection{Dificultades para la
contratación}\label{dificultades-para-la-contrataciuxf3n}

\subparagraph{Dificultades para la contratación: Sector
Construcción}\label{dificultades-para-la-contrataciuxf3n-sector-construcciuxf3n}

\subparagraph{Dificultades para la contratación: Sector
Agrícola}\label{dificultades-para-la-contrataciuxf3n-sector-agruxedcola}

\subsubsection{Educación y
experiencia}\label{educaciuxf3n-y-experiencia}

\subsubsection{Canales de reclutamiento}\label{canales-de-reclutamiento}

\subsubsection{Conocimiento de instituciones y
programas}\label{conocimiento-de-instituciones-y-programas}

\subsubsection{Apéndice A: Ocupaciones de Difícil
Cobertura}\label{apuxe9ndice-a-ocupaciones-de-difuxedcil-cobertura}

\end{document}
